% === [ IR Generation ] ========================================================

\section{IR Generation}
\label{sec:ir_generation}

One of the first decisions made when deciding to implement a compiler for the uC programming language was to target LLVM IR, rather than MIPS, as this would allow for native code generation to a wide variety of platforms, such as x86 and ARM.

A large amount of preparatory work was made before developing the IR generation phase of the compiler, in particular a library for LLVM IR was developed which implements support for parsing LLVM IR assembly files and provides an in-memory representation of LLVM IR which is capable of pretty-printing itself to LLVM IR assembly code. The design of the LLVM IR generator of the compiler has guided the design of the API for the LLVM IR library. The full API is too vast to cover here, but its documentation has been available online.

\begin{itemize}
	\item \url{https://godoc.org/github.com/llir/llvm/ir}
	\item \url{https://godoc.org/github.com/llir/llvm/ir/constant}
	\item \url{https://godoc.org/github.com/llir/llvm/ir/instruction}
	\item \url{https://godoc.org/github.com/llir/llvm/ir/types}
	\item \url{https://godoc.org/github.com/llir/llvm/ir/value}
\end{itemize}

Before diving into the design details of the IR generation phase of the compiler, a brief installation and usage introduction of the compiler is presented in section \ref{sec:installation_and_usage}, so that the interested reader may evaluate the compiler themselves on real world uC programs.

% === [ Subsections ] ==========================================================

\subsection{Installation and Usage}
\label{sec:installation_and_usage}

This section provides a brief introduction to the installation procedure and usage of the \texttt{uclang} compiler.

\subsubsection{Installation}

As a prerequisite, install a compiler for the Go programming language. See \url{https://golang.org/doc/install} for installation instructions. Make sure to configure the \texttt{GOPATH} environment variable, to specify a directory for the workspace.

Once Go has been installed and a \texttt{GOPATH} workspace has been configured, it should be possible to retrieve the source code of remote repositories and install the tools and packages contained within, using the \texttt{go} tool with the \texttt{get} command.

To generate lexers and parsers for uC source code and LLVM IR assembly the Gocc tool is used. To install Gocc, invoke the following command.

\begin{verbatim}
$ go get github.com/goccmack/gocc
\end{verbatim}

Once compiled, the \texttt{gocc} tool is placed within the \verb+$GOPATH/bin+ directory. Make sure to add this directory to the \texttt{PATH} environment variable, for subsequent \texttt{Makefile}s to work.

To simplify the generation of test cases, a regular expression search and replace tool called \texttt{sar} has been used. To download \texttt{sar}, invoke the following command.

\begin{verbatim}
$ go get github.com/mewkiz/cmd/sar
\end{verbatim}

With the prerequisites out of the way, lets take a look at how to download and install the \texttt{uclang} compiler itself. To download the compiler, and all other tools and libraries related to it, invoke the following command. You may safely ignore the warning that the \texttt{uc} directory contains no buildable Go source files.

\begin{verbatim}
$ go get -d github.com/mewmew/uc
\end{verbatim}

Now we are ready to generate the lexer and parser for uC. Traverse into the \texttt{gocc} directory of the \texttt{uc} repository and invoke \texttt{make}, i.e. invoke the following commands.

\begin{verbatim}
$ cd $GOPATH/src/github.com/mewmew/uc/gocc
$ make
\end{verbatim}

After generating the uC lexer and parser, invoke the following command to install the lexer tool, parser tool, semantic analysis tool and the uC compiler tool,

\begin{verbatim}
$ go get github.com/mewmew/uc/...
\end{verbatim}

Now, the lexer tool (\texttt{ulex}), parser tool (\texttt{uparse}), semantic analysis tool (\texttt{usem}) and the uC compiler (\texttt{uclang}) should have been compiled and installed into the \verb+$GOPATH/bin+ directory.

\subsubsection{Usage}

For usage instructions, invoke the respective commands with the \verb+-help+ flag, or refer to the online documentation.

\begin{itemize}
	\item \texttt{ulex}: \url{https://godoc.org/github.com/mewmew/uc/cmd/ulex}
	\item \texttt{uparse}: \url{https://godoc.org/github.com/mewmew/uc/cmd/uparse}
	\item \texttt{usem}: \url{https://godoc.org/github.com/mewmew/uc/cmd/usem}
	\item \texttt{uclang}: \url{https://godoc.org/github.com/mewmew/uc/cmd/uclang}
\end{itemize}

For the remainder of this section, the usage of the uC compiler \texttt{uclang} is explored and illustrated through examples.

To compile a uC source file (e.g. \texttt{foo.c}) and emit the corresponding LLVM IR assembly code to standard output, invoke the following command.

\begin{verbatim}
$ uclang foo.c
\end{verbatim}

To compile a uC source file (e.g. \texttt{foo.c}) and emit the corresponding LLVM IR assembly code to a given output file (e.g. \texttt{foo.ll}), invoke the following command.

\begin{verbatim}
$ uclang -o foo.ll foo.c
\end{verbatim}

A slightly modified version of the runtime library containing the \texttt{getint} and \texttt{putstring} functions is located at \texttt{testdata/uc.c}, and the corresponding LLVM IR assembly (compiled with Clang, since \texttt{uclang} has yet to support string literals) is located at \texttt{testdata/uc.ll}.

Using the standard \texttt{llvm-link} tool from the LLVM compiler framework, two or more LLVM IR assembly files may be linked together, or merged. This provides a useful way of linking the runtime library into compiled programs which makes use of the \texttt{getint} or \texttt{putstring} functions, such as \texttt{testdata/noisy/advanced/eval.c}. On a side node, a missing return statement has been added to the \texttt{eval} function of the \texttt{eval.c} test case source code, as the static semantic analysis checker would otherwise have terminated the compilation of the otherwise interesting test program.

To compile the \texttt{eval} test program into LLVM IR assembly, invoke the following commands.

\begin{verbatim}
$ cd $GOPATH/src/github.com/mewmew/uc
$ uclang -o out.ll testdata/noisy/advanced/eval.c
$ llvm-link -o eval.ll out.ll testdata/uc.ll
\end{verbatim}

The standard \texttt{lli} LLVM IR interpretor may then be used to invoke the resulting LLVM IR assembly file as such.

\begin{verbatim}
$ lli eval.ll
\end{verbatim}

Another option is to compile the LLVM IR assembly file to a host native binary application, by invoking the following standard command from the LLVM compiler framework.

\begin{verbatim}
$ llc -o eval.S eval.ll
$ as -o eval.o eval.S
$ ld -o eval -dynamic-linker /usr/lib/ld-linux-x86-64.so.2 /usr/lib/crt1.o
/usr/lib/crti.o /usr/lib/crtn.o eval.o -lc
\end{verbatim}

While it is possible to successfully compile and link LLVM IR assembly files using the \texttt{llc}, \texttt{as} and \texttt{ld} commands, simply invoking Clang to accomplish the same task if often far easier.

\begin{verbatim}
$ clang -o eval eval.ll
\end{verbatim}

\subsubsection{Test Cases}

To run the test cases, simply invoke the \texttt{go} tool with the \texttt{test} command and the Go import path of the package to test.

To test all packages invoke \texttt{go test github.com/mewmew/uc/...} or run the following commands.

\begin{verbatim}
$ go test github.com/mewmew/uc/token
$ go test github.com/mewmew/uc/gocc/lexer
$ go test github.com/mewmew/uc/hand/lexer
$ go test github.com/mewmew/uc/gocc/scanner
$ go test github.com/mewmew/uc/gocc/parser
$ go test github.com/mewmew/uc/sem
$ go test github.com/mewmew/uc/irgen
\end{verbatim}

\subsection{Design Decisions}

The \texttt{uclang} compiler is designed to output LLVM IR assembly which is almost identical to the LLVM IR assembly generated by Clang for a given µC program. The main benefit of this decision is to facilitate validation, as further described in section \ref{sec:irgen_validation}.

\subsubsection{Local Identifiers}

An explicit aim of LLVM IR is to remove the requirement of using other intermediate representations within the compiler middle-end \cite{osa_llvm}. To achieve this aim, the IR must be capable of supporting a wide range of optimization passes, while remaining platform independent. This is accomplished with a platform independent RISC assembly language, similar to MIPS but with support for an infinite amount of registers; as is common for Register Transfer Languages. Local IDs are assigned to keep track of these registers, and within LLVM IR, there exist a notion of unnamed local identifiers, where the ID (an integer) is inferred using a function specific counter starting from 0. The first unnamed local ID is assigned 0, the second 1, and so on, in any given function.

A core concept within LLVM IR is the notion of a \textit{value}, which is the result of a computation that may be used by other \textit{values}. Building upon this idea, an LLVM IR library has been developed which keeps track of registers, not by their explicit names, but by references to the computation which produced the register, often an arithmetic instruction. As all the key concepts within LLVM IR as represented as values, namely global variables, functions, function parameters, basic blocks, and registers assigned from instruction computations, the LLVM IR library does not keep track of names, but rather references to values, when building instructions which refer to the computed values of other instructions. This is a powerful concept, as it allows for the generation of LLVM IR, without the need to explicitly specify the names of registers. The main advantage of this is that the LLVM IR library may take the responsibility of generating consecutive sequences of local IDs, thus freeing users of the library from having to keep track of the function specific counter.

\subsubsection{Control Flow}

The control flow of a µC program is determined by the use of \texttt{if}, \texttt{else}, \texttt{while} and \texttt{return} statements and calls to functions. In our code generation, we translate the first constructs to llvm basic blocks with conditional jumping with the calls and returns staying essentially the same.

%\subsubsection{Implicit Conversion}
%TODO: alex

%% Skipping this section fow now.

%\subsection{Implementation}

%\subsubsection{Control Flow}
%TODO: alex

%\subsubsection{Implicit Conversion}
%TODO: alex

\subsection{Validation}
\label{sec:irgen_validation}

As the LLVM IR generated by the \texttt{uclang} compiler has been intentionally kept very close to the LLVM IR generated by Clang for a given uC program, testing the implementation of the IR generator becomes almost trivial. The test cases simply use Clang to generate LLVM IR from input C source files, each covering a specific language features, such as while loops or array index expressions. Since Clang outputs a lot of metadata, which is not strictly related to the semantics of the program, a post-processing script has been developed which removes any unnecessary metadata, function attributes, comments or other LLVM IR language constructs which are not required to run the test programs. The post-processed LLVM IR assembly output from Clang is then compared against the LLVM IR generated by the \texttt{uclang} IR generation algorithm, and the test case passes if the outputs are identical and it fails otherwise.

To the best of our knowledge, the LLVM IR generated by \texttt{uclang} should be valid for all possible uC programs, with one exception array index expressions and array references in functions taking arrays as input parameters. That being said, we have discovered a wide range of issues with the generation algorithm while developing it, and have added specific test cases to ensure that fixed bugs are not reintroduced at a later point of development.

To run the test cases of the IR generation library, invoke the following command.

\begin{verbatim}
$ go test github.com/mewmew/uc/irgen
\end{verbatim}

The input source files used to validate the IR generation algorithm may be located in the \texttt{testdata/extra/irgen} directory of the \texttt{uc} repository.

\subsubsection{Example Output}

The listing example illustrates the LLVM IR output (see listing \ref{lst:r06.ll}) generated by \texttt{uclang} when compiling the Fibonacci test case \texttt{testdata/quiet/rtl/r06.c} (see listing \ref{lst:quiet/rtl/r06.c}). Note, the \texttt{r06.c} source file has been slightly modified, to commend out the preprocessing include directive and to include a return statement from main, returning the value calculated by the \texttt{fib} function as a status code from the progam.

\lstinputlisting[style=c,language=C,caption=quiet/rtl/r06.c,label=lst:quiet/rtl/r06.c]{inc/sections/6_ir_generation/r06.c}

\lstinputlisting[style=c,language=llvm,caption=r06.ll,label=lst:r06.ll]{inc/sections/6_ir_generation/r06.ll}

To compile test \texttt{r06.c} source file and run the resulting LLVM IR assembly code, invoke the following commands.

\begin{verbatim}
$ uclang -o r06.ll r06.c
$ lli r06.ll ; echo $?
\end{verbatim}

Executing the above command will print the status code \texttt{120} to standard output, thus confirming that the fifth Fibonacci number was indeed calculated correctly.

%\subsubsection{Correct RTL Test Cases}

%For listings, see Appendix~\ref{app:irgen/correct}: \nameref{app:irgen/correct}:
%\input{inc/uclang/quiet/rtl/listref.tex}

%\subsubsection{Extra Test Cases}

%For listings, see Appendix~\ref{app:irgen/extra}: \nameref{app:irgen/extra}:
%\input{inc/uclang/noisy/advanced/listref.tex}
%\input{inc/uclang/extra/irgen/listref.tex}

