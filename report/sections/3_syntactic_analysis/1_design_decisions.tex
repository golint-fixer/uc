\subsection{Design Decisions}

foo

\subsubsection{Unified Types}

To simplify the design of the syntactic analysis phase, types have been given a unified representation in the grammar. As a consequence of this design decision, certain invalid type declarations are syntactically valid, such as \texttt{void x[10];}. To facilitate separation of concern between components of the compiler, proper type checking has intentially been postponed to the semantic analysis phase\footnote{Postpone type checking to the semantic analysis phase: \url{https://github.com/mewmew/uc/issues/33}}.

\subsubsection{Unified Declarations}

To further simplify the grammar, variable and function declarations have been given a unified representation in the grammar\footnote{Evaluate merging TopLevelDecl with Decl to simplify grammar: \url{https://github.com/mewmew/uc/issues/38}}. As a consequence of this design decision, nested function declarations are syntactially valid (see listing \ref{fig:nested_func_decl}), but not necessarily semantically valid.

\begin{lstlisting}[language=C,style=c,caption={\label{fig:nested_func_decl}Nested function declarations.}]
int add(int a, int b) {
	// Nested function declarations are syntactially valid.
	int nested(void) {
		return a + b;
	}
	return nested();
}
\end{lstlisting}

The static checker of the semantic analysis phase will ensure that functions contain no nested function declarations, unless the relevant GNU extension has been enabled\footnote{Add support for nested functions (GNU extension): \url{https://github.com/mewmew/uc/issues/43}}.

\subsubsection{Ignored Comments}

% Having comment tokens emitted by the lexer into the stream of tokens, presents an issue when parsing source files; as the current grammar does not handle comments within its production rules.
%
% Two approaches for resolving this issue have been identified
%
%     Add comments to the production rules.
%         Pros: makes it easier to write tools such as indent and doxygen which requires preservation of position accurate comments, and hopefully attached to their corresponding node within the AST.
%         Cons: makes the grammar much more verbose, as a comment can exist virtually between any two given tokens.
%     Ignore comments when lexing the source file.
%         Pros: simplifies the grammar.
%         Cons: does not facilitate the development of source code rewriting and documentation generation tools.
%
% For the time being, the second approach has been implemented. This decision may be revisited once the parser has matured to correctly handle the syntactic elements of uC.

% The second approach has been implemented in commit 1488a8a, closing for now. Re-open this topic in the future, when trying to tackle the first approach.

foo

\footnote{Production rules made more complex by comment tokens: \url{https://github.com/mewmew/uc/issues/30}}
