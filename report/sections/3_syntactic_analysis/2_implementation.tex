\subsection{Implementation}

The parser was generated using Gocc from a BNF grammar of the uC language, the listing of which is presented in Appendix~\ref{app:parser/gocc}. The precedence and associativity of binary operators have been cross-referenced against §6.5 of the C11 specification \cite{c11_spec}.

\subsubsection{Precedence of Binary Operators}

Precedence of binary operators is implemented in the BNF grammar through a tree-like structure of production rules (actually a graph structure, as nodes may be self-referential and refer to parental nodes), having operators with lower precedence near the root of the tree and operators with higher precedence near the leafs of the tree.

An example of the production rules used to assign a higher precedence to multiplicative operators than additive operators is presented in listing \ref{lst:precedence}.

\begin{lstlisting}[language=go,style=go,caption={\label{lst:precedence}Precedence of binary expressions.}]
// Left-associative binary expressions with precedence 12.
//
//    12L: + -
Expr12L
	: Expr13L
	| Expr12L "+" Expr13L
	| Expr12L "-" Expr13L
;

// Left-associative binary expressions with precedence 13.
//
//    13L: * /
Expr13L
	: Expr14
	| Expr13L "*" Expr14
	| Expr13L "/" Expr14
;
\end{lstlisting}

\subsubsection{Associativity of Binary Operators}

The implementation of left- and right-associativity of binary operators relies on the same tree-like structure as the implementation of operator precedence. The non-terminal node on the right side of a right-associative binary operator (e.g. \texttt{Expr2R} of \texttt{Expr5L ``='' Expr2R}) is self-referential, while the non-terminal node on the left side of the operator (e.g. \texttt{Expr5L} of \texttt{Expr5L ``='' Expr2R}) refers to a node closer to the leaf nodes of the tree. Left-associativity is defined analogously, with a self-referential left side node.

An example of the production rules used to define right-associativity to assignment operators and left-associativity to logical AND operators is presented in listing \ref{lst:associativity}.

\begin{lstlisting}[language=go,style=go,caption={\label{lst:associativity}Associativity of binary expressions.}]
// Right-associative binary expressions with precedence 2.
//
//    2R: =
Expr2R
	: Expr5L
	// Right-associative.
	| Expr5L "=" Expr2R
;

// Left-associative binary expressions with precedence 5.
//
//    5L: &&
Expr5L
	: Expr9L
	| Expr5L "&&" Expr9L
;
\end{lstlisting}

\subsubsection{Syntax Tree Representation}

See Appendix~\ref{app:parser/ast_doc}.

% You should also describe the representation of syntax trees and list the different kinds of nodes.

% TODO: Add syntax tree representation.

\subsubsection{Dangling Else}

The dangling else problem is the canonical example of a shift-reduce conflict for LR grammars, and it is present in the µC grammar presented in Appendix~\ref{app:parser/gocc}. When generating a parser from this grammar, Gocc was capable of identifying the shift-reduce conflict introduce by the dangling else problem. Furthermore, Gocc provided an option for automatically resolving shift-reduce conflicts\footnote{Dangling else: \url{https://github.com/mewmew/uc/issues/31}} by applying \textit{maximum-munch}, as further described by the following extract from the Gocc user guide \cite{gocc_user_guide}.

\begin{quote}
	\begin{lstlisting}[language=go,style=go,caption={Grammar with shift-reduce conflict.}]
	Stmt
		: "if" Expr "then" Stmt
		| "if" Expr "then" Stmt "else" Stmt
	;
	\end{lstlisting}

	\textit{``When automatic LR(1) conflict resolution is selected by the -a option, gocc resolves this conflict in the same way as specified in the C language specification: by shifting and parsing the longest valid production (maximal-munch). This means recognising the else-statement as part of the second if.''}
\end{quote}
