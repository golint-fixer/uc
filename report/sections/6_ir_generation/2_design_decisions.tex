\subsection{Design Decisions}

The \texttt{uclang} compiler is designed to output LLVM IR assembly which is almost identical to the LLVM IR assembly generated by Clang for a given µC program. The main benefit of which is to facilitate validation, as further described in section \ref{sec:irgen_validation}.

\subsubsection{Local Identifiers}

An explicit aim of LLVM IR is to remove the requirement of using other intermediate representations within the compiler middle-end \ref{osa_llvm}. To achive this aim, the IR must be capable of supporting a wide range of optimization passes, while remaining platform independent. This is accomplished with a platform independent RISC assembly language, similar to MIPS but with support for an infinite amount of registers; as is common for Register Transfer Languages. Local IDs are assigned to keep track of these registers, and within LLVM IR, there exist a notion of unnamed local identifiers, where the ID (an integer) is inferred using a function specific counter starting from 0. The first unnamed local ID is assigned 0, the second 1, and so on, in any given function.

A core concept within LLVM IR is the notion of a \textit{value}, which is the result of a computation that may be used by other \textit{values}. Building upon this idea, an LLVM IR library has been developed which keeps track of registers, not by their explicit names, but by references to the computation which produced the register, often an arithmetic instruction. As all the key concepts within LLVM IR as represented as values, namely global variables, functions, function parameters, basic blocks, and registers assigned from instruction computations, the LLVM IR library does not keep track of names, but rather references to values, when building instructions which refer to the computed values of other instructions. This is a powerful concept, as it allows for the generation of LLVM IR, without the need to explicity specify the names of registers. The main advantage of this is that the LLVM IR library may take the responsibility of generating consecutive sequences of local IDs, thus freeing users of the library from having to keep track of the function specific counter.

\subsubsection{Control Flow}

The control flow of a µC program is determined by the use of \texttt{if}, \texttt{else}, \texttt{while} and \texttt{return} statements and calls to functions. In our code generation, we translate the first constructs to llvm basic blocks with conditional jumping with the calls and returns staying essentially the same.

\subsubsection{Implicit Conversion}
TODO: alex
