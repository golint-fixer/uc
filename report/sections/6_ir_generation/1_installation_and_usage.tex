\subsection{Installation and Usage}
\label{sec:installation_and_usage}

This section provides a brief introduction to the installation procedure and usage of the \texttt{uclang} compiler.

\subsubsection{Installation}

As a prerequisite, install a compiler for the Go programming language. See \url{https://golang.org/doc/install} for installation instructions. Make sure to configure the \texttt{GOPATH} environment variable, to specify a directory for the workspace.

Once Go has been installed and a \texttt{GOPATH} workspace has been configured, it should be possible to retrieve the source code of remote repositories and install the tools and packages contained within, using the \texttt{go} tool with the \texttt{get} command.

To generate lexers and parsers for uC source code and LLVM IR assembly the Gocc tool is used. To install Gocc, invoke the following command.

\begin{verbatim}
$ go get github.com/goccmack/gocc
\end{verbatim}

Once compiled, the \texttt{gocc} tool is placed within the \verb+$GOPATH/bin+ directory. Make sure to add this directory to the \texttt{PATH} environment variable, for subsequent \texttt{Makefile}s to work.

To simplify the generation of test cases, a regular expression search and replace tool called \texttt{sar} has been used. To download \texttt{sar}, invoke the following command.

\begin{verbatim}
$ go get github.com/mewkiz/cmd/sar
\end{verbatim}

With the prerequisites out of the way, lets take a look at how to download and install the \texttt{uclang} compiler itself. To download the compiler, and all other tools and libraries related to it, invoke the following command. You may safely ignore the warning that the \texttt{uc} directory contains no buildable Go source files.

\begin{verbatim}
$ go get -d github.com/mewmew/uc
\end{verbatim}

Now we are ready to generate the lexer and parser for uC. Traverse into the \texttt{gocc} directory of the \texttt{uc} repository and invoke \texttt{make}, i.e. invoke the following commands.

\begin{verbatim}
$ cd $GOPATH/src/github.com/mewmew/uc/gocc
$ make
\end{verbatim}

After generating the uC lexer and parser, invoke the following command to install the lexer tool, parser tool, semantic analysis tool and the uC compiler tool,

\begin{verbatim}
$ go get github.com/mewmew/uc/...
\end{verbatim}

Now, the lexer tool (\texttt{ulex}), parser tool (\texttt{uparse}), semantic analysis tool (\texttt{usem}) and the uC compiler (\texttt{uclang}) should have been compiled and installed into the \verb+$GOPATH/bin+ directory.

\subsubsection{Usage}

For usage instructions, invoke the respective commands with the \verb+-help+ flag, or refer to the online documentation.

\begin{itemize}
	\item \texttt{ulex}: \url{https://godoc.org/github.com/mewmew/uc/cmd/ulex}
	\item \texttt{uparse}: \url{https://godoc.org/github.com/mewmew/uc/cmd/uparse}
	\item \texttt{usem}: \url{https://godoc.org/github.com/mewmew/uc/cmd/usem}
	\item \texttt{uclang}: \url{https://godoc.org/github.com/mewmew/uc/cmd/uclang}
\end{itemize}

For the remainder of this section, the usage of the uC compiler \texttt{uclang} is explored and illustrated through examples.

To compile a uC source file (e.g. \texttt{foo.c}) and emit the corresponding LLVM IR assembly code to standard output, invoke the following command.

\begin{verbatim}
$ uclang foo.c
\end{verbatim}

To compile a uC source file (e.g. \texttt{foo.c}) and emit the corresponding LLVM IR assembly code to a given output file (e.g. \texttt{foo.ll}), invoke the following command.

\begin{verbatim}
$ uclang -o foo.ll foo.c
\end{verbatim}

A slightly modified version of the runtime library containing the \texttt{getint} and \texttt{putstring} functions is located at \texttt{testdata/uc.c}, and the corresponding LLVM IR assembly (compiled with Clang, since \texttt{uclang} has yet to support string literals) is located at \texttt{testdata/uc.ll}.

Using the standard \texttt{llvm-link} tool from the LLVM compiler framework, two or more LLVM IR assembly files may be linked together, or merged. This provides a useful way of linking the runtime library into compiled programs which makes use of the \texttt{getint} or \texttt{putstring} functions, such as \texttt{testdata/noisy/advanced/eval.c}. On a side node, a missing return statement has been added to the \texttt{eval} function of the \texttt{eval.c} test case source code, as the static semantic analysis checker would otherwise have terminated the compilation of the otherwise interesting test program.

To compile the \texttt{eval} test program into LLVM IR assembly, invoke the following commands.

\begin{verbatim}
$ cd $GOPATH/src/github.com/mewmew/uc
$ uclang -o out.ll testdata/noisy/advanced/eval.c
$ llvm-link -o eval.ll out.ll testdata/uc.ll
\end{verbatim}

The standard \texttt{lli} LLVM IR interpretor may then be used to invoke the resulting LLVM IR assembly file as such.

\begin{verbatim}
$ lli eval.ll
\end{verbatim}

Another option is to compile the LLVM IR assembly file to a host native binary application, by invoking the following standard command from the LLVM compiler framework.

\begin{verbatim}
$ llc -o eval.S eval.ll
$ as -o eval.o eval.S
$ ld -o eval -dynamic-linker /usr/lib/ld-linux-x86-64.so.2 /usr/lib/crt1.o
/usr/lib/crti.o /usr/lib/crtn.o eval.o -lc
\end{verbatim}

While it is possible to successfully compile and link LLVM IR assembly files using the \texttt{llc}, \texttt{as} and \texttt{ld} commands, simply invoking Clang to accomplish the same task if often far easier.

\begin{verbatim}
$ clang -o eval eval.ll
\end{verbatim}

\subsubsection{Test Cases}

To run the test cases, simply invoke the \texttt{go} tool with the \texttt{test} command and the Go import path of the package to test.

To test all packages invoke \texttt{go test github.com/mewmew/uc/...} or run the following commands.

\begin{verbatim}
$ go test github.com/mewmew/uc/token
$ go test github.com/mewmew/uc/gocc/lexer
$ go test github.com/mewmew/uc/hand/lexer
$ go test github.com/mewmew/uc/gocc/scanner
$ go test github.com/mewmew/uc/gocc/parser
$ go test github.com/mewmew/uc/sem
$ go test github.com/mewmew/uc/irgen
\end{verbatim}
